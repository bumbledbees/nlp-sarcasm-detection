\documentclass[11pt]{article}

% packages from hadi's template
\usepackage{bbm}
\usepackage{amsmath}
\usepackage{amssymb}
\usepackage{amsthm}
% \usepackage{chngpage}
\usepackage{fancyhdr}
\usepackage[margin=.7in]{geometry}
\usepackage{graphicx}
\usepackage{hyperref}
% \usepackage{lscape}
\usepackage{mathpazo}
\usepackage{stmaryrd}
% \usepackage{subfigure}
\usepackage{url}

% packages from nam's template
\usepackage{authblk}
\usepackage{amsfonts}
% \usepackage{biblatex}
\usepackage{float}
\usepackage[utf8]{inputenc}
\usepackage{siunitx}
\usepackage{subcaption}
\usepackage[nottoc,numbib]{tocbibind}

\pagestyle{fancy}

\newcommand{\email}[1]{\texttt{\href{mailto:#1}{#1}}}

\lhead{\textbf{Project Proposal}}
\rhead{\textbf{COMP.4420/5420, UMASS Lowell}}

\def\proptitle{COMP4420 Project Proposal: Sarcasm Detection}
\def\propauthors{Bui, Nam (\#01963609), 
                 Conners, Riley (\#01943861), 
                 Zuk, Sam (\#01642608)}

\begin{document}

\begin{center}
    \textbf{\Large{\proptitle}} \\
    \textbf{\underline{\propauthors}}
\end{center}

\bigskip

\section{Introduction}

Sarcasm is a feature of natural language that is notoriously difficult to
define and identify in both the spoken and written word. The assumption that a
statement will be recognized as sarcastic is typically contingent upon the
listener/reader knowing some outside piece of contextual information
beforehand. However, this external information isn't always known, and even
when it is, the relationship between it and the statement at hand may not
always be clear. When this happens, the meaning can be obscured as a result,
often leading to avoidable scenarios involving miscommunication.

Recognizing sarcasm typically involves picking up on subtle cues and nuance
that can be difficult to identify. This can often pose a challenge for
populations who encounter greater difficulty when processing certain aspects of
a language. For example, someone trying to interpret a language they don't
speak natively will likely have to expend more mental effort to parse out
meaning from words, which in turn makes it more difficult to pick up on nuance,
including sarcasm. Being unfamiliar with the cultural norms, idioms, etc. that
inform the established meaning of the locally spoken language can also be a
source of confusion. In addition, many neurodivergent people, in particular
those with autism, can struggle to recognize and/or communicate certain social
cues in conversation due to differences between their cognitive experience of
language and what is expected of them.

Finally, there are unique challenges faced in detecting sarcasm in the written
word. It is often possible in practice to infer a statement is sarcastic, even
without necessarily having the context to understand \textit{why} by listening
to changes in the tone of the speaker. However, when translated into the
written word, some or all of this information is lost, making sarcasm even more
difficult to detect when only text is given. With the Internet now being
extremely important to modern infrastructure, and with text being the
predominant medium for online communication, this problem has become
increasingly apparent over the years. This project shall explore and contrast
different approaches to disambiguating sarcasm by applying concepts from the
fields of computational linguistics and machine learning.

\section{Dataset}

The dataset we plan on using for this project is a collection of 28,619 tagged
newspaper headlines-- of which 13,635 are from the satirical publication
\textit{The Onion}, the other 14,984 being from the non-satirical publication
\textit{The Huffington Post} \cite{misra2023Sarcasm}.

% Structure of data in dataset
The data for each headline is grouped into three characteristics. There is a
boolean for marking if the headline is sarcastic (1) or not (0), the text of
the headline, and a link to the article so that other information can be
gathered if desired.

% When and how dataset was collected
% Headlines collected from The Onion (sarcastic) and HuffPost (sincere)
Data was collected from two news sources, namely the Onion, who focuses on
sarcastic news parodies, and The Huffington Post, who is a sincere media
company. This data was then updated in July 2019 to include more recent
headlines.

% Headlines from sources do not intersect (self-contained)
This dataset has advantages over text that could be found on social media
platforms because news text is formal in nature. This means there are less
words outside of the word2vec vocabulary, less spelling mistakes, and little to
no slang usage. Also, because The Onion is openly sarcastic by design, there is
no ambiguity regarding if labels are correct.

% Limitation of dataset: Only two news sources
% Additionally, sarcastic headlines from The Onion are obviously sarcastic,
% so more subtle sarcasm is not captured in the dataset
However, there are downsides to news headline data. In this case, there are
only two news sources being used, and the model could pick up on writing styles
or other details instead of sarcasm. There is another potential issue that
stems from The Onion's obvious use of sarcasm. In more nuanced cases where
sarcasm is more subtle, a model could do poorly.

\section{Evaluation Method}

% Will be interesting to do some EDA
Since the dataset was created in 2016 during a period of political turmoil,
there may be some bias in the data. It will be interesting to see what words
are most strongly correlated with sarcastic headlines.

% News tends to have a lot of named entities,
% So it may be useful to have some NER
Additionally, news headlines usually have a lot of proper nouns, so it may help
to use named entity recognition when encoding the headlines.

% Models to test:
% "Traditional" models like Naive Bayes
%   (Nam: IIRC I got >80% acc w/ NB and one-hot encoding in 2020)
% Context-independent NN like DAN
% Context-sensitive RNN w/ attention has been shown to perform well on this task
Sentiment analysis is a core natural language processing task, so there is a
lot of data available on what types of models are effective. We plan on using
several for this project. Naive Bayes classifiers are lightweight models that
have traditionally been used in sentiment analysis. Deep averaging networks are
able to leverage the universal approximation properties of neural networks, but
are lightweight since they don't capture context. In recent years, recurrent
neural networks have gained popularity due to their ability to capture context
with the attention mechanism. Since news headlines are often one or two
sentences, there is not much need to capture long distance dependencies.

% Since it's classification, F1 is a decent metric
Since the task at hand is binary classification, precision, recall, and F1 are
good metrics to use. Accuracy will also be used to compare findings to results
from Misra et al \cite{misra2023Sarcasm}.

\section{Timeline}
% Split data into training, dev, and test (70/10/20)
First, the dataset will be split into training, development, and test sets with
a 70/10/20 split.

% Baseline: Naive Bayes w/ one-hot encoding
For the baseline model, we will use a Naive-Bayes with one-hot encoding.

% Fine-tune word2vec for dataset
After training and evaluating the baseline model, we would like to fine-tune
word2vec to the headline-specific words.

% DAN + RNN w/ LSTM
Once the word embeddings have been fine tuned, we would like to train and
evaluate either a DAN or LSTM model.

\bibliographystyle{plain}
\bibliography{ref}

\end{document}
