\documentclass[11pt]{article}

\usepackage{authblk}
\usepackage{hyperref}
\usepackage[utf8]{inputenc}
\usepackage{amsmath}
\usepackage{amsfonts}
\usepackage{amssymb}
\usepackage{siunitx}
\usepackage{graphicx}
\usepackage{subcaption}
\usepackage{float}
\usepackage[nottoc,numbib]{tocbibind}
\usepackage{biblatex}

\bibliography{ref.bib}

\newcommand{\email}[1]{\texttt{\href{mailto:#1}{#1}}}

\title{COMP4420 Project Proposal: Sarcasm Detection}
\author{
    Bui, Nam \\
    \email{nam\_bui@student.uml.edu}
    \and
    Sam, Zuk \\
    \email {samuel\_zuk@student.uml.edu}
    % \and
    % Put names here
}

\begin{document}

\maketitle

\section{Introduction}

\par{Sarcasm is a feature of natural language that is notoriously difficult to
define and identify in both the spoken and written word. The assumption that a
statement will be recognized as sarcastic is typically contingent upon the
listener/reader knowing some outside piece of contextual information
beforehand. However, this external information isn't always known, and even
when it is, the relationship between it and the statement at hand may not
always be clear. When this happens, the meaning can be obscured as a result,
often leading to avoidable scenarios involving miscommunication.}

\par{Recognizing sarcasm typically involves picking up on subtle cues and
nuance that can be difficult to identify. This can often pose a challenge for
populations who encounter greater difficulty when processing certain aspects of
a language. For example, someone trying to interpret a language they don't
speak natively will likely have to expend more mental effort to parse out
meaning from words, which in turn makes it more difficult to pick up on nuance,
including sarcasm. Being unfamiliar with the cultural norms, idioms, etc. that
inform the established meaning of the locally spoken language can also be a
source of confusion. In addition, many neurodivergent people, in particular
those with autism, can struggle to recognize and/or communicate certain social
cues in conversation due to differences between their cognitive experience of
language and what is expected of them.}

\par{Finally, there are unique challenges faced in detecting sarcasm in the
written word. It is often possible in practice to infer a statement is
sarcastic, even without necessarily having the context to understand
\textit{why} by listening to changes in the tone of the speaker. However, when
translated into the written word, some or all of this information is lost,
making sarcasm even more difficult to detect when only text is given. With the
Internet now being extremely important to modern infrastructure, and with text
being the predominant medium for online communication, this problem has become
increasingly apparent over the years. This project shall explore and contrast
different approaches to disambiguating sarcasm by applying concepts from the
fields of computational linguistics and machine learning.}

% Sentiment analysis

% The difficulty of determining sarcasm
% e.g. sentence in one context may be sarcastic while in another context may be serious

\section{Dataset}

\par{The dataset we plan on using for this project is a collection of 28,619
tagged newspaper headlines-- of which 13,635 are from the satirical publication
\textit{The Onion}, the other 14,984 being from the non-satirical publication
\textit{The Huffington Post} \cite{misra2023Sarcasm}.}

% Structure of data in dataset
% When and how dataset was collected

% Headlines collected from TheOnion (sarcastic) and Huffington Post (not-sarcastic)
% Headlines from sources do not intersect (self-contained)
% Limitation of dataset: Only two news sources
% Additionally, sarcastic headlines from TheOnion are obviously sarcastic,
% so more subtle sarcasm is not captured in the dataset

\section{Evaluation Method}

% Will be interesting to do some EDA
\par{Since the dataset was created in 2016 during a period of political
turmoil, there may be some bias in the data. It will be interesting to see what
words are most strongly correlated with sarcastic headlines.}

% News tends to have a lot of named entities,
% So it may be useful to have some NER
\par{Additionally, news headlines usually have a lot of proper nouns, so it may
help to use named entity recognition when encoding the headlines.}

% Models to test:
% "Traditional" models like Naive Bayes (IIRC I got >80% acc w/ NB and one-hot encoding in 2020)
% Context-independent NN like DAN
% Context-sensitive RNN w/ attention has been shown to perform well on this task
\par{Sentiment analysis is a core natural language processing task, so there is
a lot of data available on what types of models are effective. We plan on using
several for this project. Naive Bayes classifiers are lightweight models that
have traditionally been used in sentiment analysis. Deep averaging networks are
able to leverage the universal approximation properties of neural networks, but
are lightweight since they don't capture context. In recent years, recurrent
neural networks have gained popularity due to their ability to capture context
with the attention mechanism. Since news headlines are often one or two
sentences, there is not much need to capture long distance dependencies.}

% Since it's classification, F1 is a decent metric
\par{Since the task at hand is binary classification, precision, recall, and F1
are good metrics to use. Accuracy will also be used to compare findings to
results from Misra et al \cite{misra2023Sarcasm}.}

\printbibliography

\end{document}
